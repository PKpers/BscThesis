\chapter{Conclusion}
\label{sec:org11ceed0}
In this work, we present a study of the interplay between multivariate and single variate classification techniques, and energy scale uncertainties in the search for both heavy and light diobject resonances at the LHC. Moreover, we consider a plethora of cases of uncertainties in the energy scale, in the analysis of the performance of a BDT model, and a fit-based analysis.

Overall, we observe that the BDT model is more robust when systematic uncertainties are present, while delivering better performance in terms of significance. Such behavior is related to the fact that the classifier learns (in a rather unphysical way) to classify the signal and background using features that are not affected by energy scale uncertainties.

This does not mean that a fit-based analysis is of no use, as there are ways to enhance its performance. Specifically, looking at the feature space discussed in Sections \ref{sec:Training} and \ref{sec:LightTraining}, it is rather evident that the feature that helps the most with the distinction between the signal and background, while being invariant to energy scale uncertainties, is the azimuthal angle difference of the two produced Xs. Therefore, with a more careful event selection, one can reject a significant portion of the background by taking into consideration the \(\Delta \phi\) of the pair, which is expected to result in a better performance of the fit-based approach.

Nevertheless, through the thorough analysis presented in this thesis, we conclude that when systematic uncertainties are introduced to the problem, there will be cases where the signal mass will be smeared away completely, making the use of univariate classification techniques challenging. In such cases, it is expected that a multivariate classification method, like a BDT model, will be able to perform the task and deliver decent performance.

Finally, to train the BDT model, we had to find the feature space that exploits the differences between signal and background in the most efficient way. In our case, the best features were the Pt of the particles, \(\Delta\Phi\), \(\Delta\eta\) and \(\Delta R\). As discussed, the most helpful geometric feature for the distinction of the particles is the difference in the azimuthal angle of the pair. This difference in \(\Delta \phi\) is inherited from the MC samples we used for the signal and background (background comes from the Drell-Yan process, and signal from W\(\Phi\) to ll), be that as it may, the argument that the resulting dataset can be treated as a generic diobject production still holds. For if the parent MC samples were different, the best features would also have been different, but the rest of the analysis would have been the same, a process-agnostic search.
